\documentclass[11pt,a4paper]{article}
\usepackage[margin=0.75in]{geometry}
\usepackage{tabularx}
\usepackage{multirow}
\usepackage{graphicx}
\usepackage{helvet}
\usepackage{lastpage}
\renewcommand{\familydefault}{\sfdefault}
\setlength\parindent{0pt}
\begin{document}

%-------------------------------------------------
% HEADER
%-------------------------------------------------
\noindent
\begin{tabularx}{\linewidth}{|p{0.20\linewidth}|p{0.55\linewidth}|p{0.20\linewidth}|}
\hline
\centering\textbf{LOGO} &
\centering\begin{tabular}{c}
\textbf{Materials \& Metallurgy Group}\\[2mm]
\textbf{Material Specification}\\
\textbf{[Material Name]}
\end{tabular} &
\begin{tabular}{|p{0.45\linewidth}|p{0.45\linewidth}|}
\hline
\textbf{Doc. No.} & \textbf{[Doc Number]} \\ \hline
\textbf{Sheet}   & \textbf{Page \thepage\ of \pageref{LastPage}} \\ \hline
\end{tabular} \\
\hline
\end{tabularx}
\vspace{1.5em}

%-------------------------------------------------
% 1. SCOPE
%-------------------------------------------------
\vspace{1em}\noindent\textbf{1. SCOPE:}
The present document outlines the teaching objectives, content coverage and reference material for the
Semester‑III course “Regulation 2021”.  The course is designed to provide students with a comprehensive
understanding of phase diagrams, heat‑treatment processes, alloy classifications, non‑metallic material
types, and testing methods.  Upon successful completion, students will be able to:

\begin{enumerate}
\item Explain alloys and phase diagrams, including the iron‑iron‑carbide system and steel classification.
\item Describe isothermal transformation, continuous‑cooling diagrams and the principal heat‑treatment
      processes.
\item Discuss the influence of alloying elements on ferrous and non‑ferrous metals.
\item Summarise the properties and applications of polymers, ceramics and composites.
\item Identify and apply the principal mechanical‑testing techniques.
\end{enumerate}
\vspace{1em}

%-------------------------------------------------
% 2. SPECIFICATIONS
%-------------------------------------------------
\vspace{1em}\noindent\textbf{2. SPECIFICATIONS:}
\begin{center}
\begin{tabular}{|c|c|c|c|}
\hline
\textbf{Code} & \textbf{Colour} & \textbf{Equivalent Spec.} & \textbf{As Per Standard} \\
\hline
--- & --- & --- & --- \\
\hline
\end{tabular}
\end{center}
\vspace{1em}

%-------------------------------------------------
% 3. CHEMICAL COMPOSITION
%-------------------------------------------------
\vspace{1em}\noindent\textbf{3. CHEMICAL COMPOSITION:}
\begin{center}
\begin{tabular}{|c|c|c|c|c|c|c|c|c|}
\hline
\textbf{Element} & C & Si & Mn & P & S & Cr & Ni & Mo & Others \\
\hline
\textbf{Minimum (\%)} & --- & --- & --- & --- & --- & --- & --- & --- & --- \\
\hline
\textbf{Maximum (\%)} & --- & --- & --- & --- & --- & --- & --- & --- & --- \\
\hline
\end{tabular}
\end{center}
\noindent\footnotesize{*No quantitative chemical data are applicable for this course specification.}
\vspace{1em}

%-------------------------------------------------
% 4. MECHANICAL PROPERTIES
%-------------------------------------------------
\vspace{1em}\noindent\textbf{4. MECHANICAL PROPERTIES:}
\begin{center}
\begin{tabular}{|c|c|c|}
\hline
\textbf{Property} & \textbf{Minimum} & \textbf{Maximum} \\
\hline
Hardness (as supplied) & --- & --- \\
\hline
Tensile Strength & --- & --- \\
\hline
Yield Strength & --- & --- \\
\hline
Elongation (\%) & --- & --- \\
\hline
Impact Energy & --- & --- \\
\hline
\end{tabular}
\end{center}
\vspace{1em}

%-------------------------------------------------
% 5. METALLURGICAL PROPERTIES
%-------------------------------------------------
\vspace{1em}\noindent\textbf{5. METALLURGICAL PROPERTIES:}
\begin{center}
\begin{tabular}{|p{0.12\linewidth}|p{0.80\linewidth}|}
\hline
\textbf{5.1} & Understanding of phase‑diagram construction and interpretation (Fe‑FeC, binary, ternary). \\
\hline
\textbf{5.2} & Knowledge of solid‑solution, substitutional and interstitial alloying mechanisms. \\
\hline
\textbf{5.3} & Ability to identify and classify steel and cast‑iron microstructures. \\
\hline
\textbf{5.4} & Familiarity with heat‑treatment processes (annealing, normalising, hardening, tempering, case‑hardening). \\
\hline
\textbf{5.5} & Comprehension of continuous‑cooling transformation (CCT) and isothermal transformation (IT) diagrams. \\
\hline
\textbf{5.6} & Insight into alloying‑element effects on mechanical properties (Mn, Si, Cr, Mo, Ni, V, Ti, W). \\
\hline
\textbf{5.7} & Overview of non‑metallic materials (polymers, ceramics, composites) and their engineering applications. \\
\hline
\textbf{5.8} & Proficiency in mechanical‑testing methods (hardness, impact, fatigue, creep). \\
\hline
\textbf{5.9} & Understanding of inclusion rating systems (e.g., ASTM E45) and their relevance to material performance. \\
\hline
\end{tabular}
\end{center}
\vspace{1em}

%-------------------------------------------------
% REFERENCES
%-------------------------------------------------
\vspace{1em}\noindent\textbf{References}
\begin{enumerate}
\item Kenneth G. Budinski and Michael K. Budinski, \emph{Engineering Materials}, Prentice Hall of India, 9th ed., 2018.
\item Sydney H. Avner, \emph{Introduction to Physical Metallurgy}, McGraw‑Hill, 1994.
\item A. Alavudeen, N. Venkateshwaran and J. T. Winowlin Jappes, \emph{A Textbook of Engineering Materials and Metallurgy}, Laxmi Publications, 2006.
\item Amandeep S. Wadhwa and Harvinder S. Dhaliwal, \emph{A Textbook of Engineering Material and Metallurgy}, University Sciences Press, 2008.
\item G. S. Upadhyay and Anish Upadhyay, \emph{Materials Science and Engineering}, Viva Books, 2020.
\item R. V. Raghavan, \emph{Materials Science and Engineering}, Prentice Hall of India, 6th ed., 2019.
\item W. D. Callister, \emph{Materials Science and Engineering}, Wiley India, 2nd ed., 2019.
\end{enumerate}

\end{document}