\documentclass[11pt,a4paper]{article}
\usepackage[margin=0.75in]{geometry}
\usepackage[table]{xcolor}
\usepackage{tabularx, multirow, multicol, graphicx, helvet, siunitx, tikz, pgfplots}
\usepackage[most]{tcolorbox}
\renewcommand{\familydefault}{\sfdefault}

%--- colour palette -------------------------------------------------
\definecolor{CorpBlue}{HTML}{003366}
\definecolor{CorpLightBlue}{HTML}{E6F0FA}
\definecolor{CorpGrey}{HTML}{F4F4F4}
\definecolor{AccentRed}{HTML}{CC0000}

%--- pgfplots settings -----------------------------------------------
\pgfplotsset{
    compat=1.18,
    grid style={dashed,gray!30}
}

\begin{document}

%=== CORPORATE HEADER =================================================
\noindent
\begin{tabular}{|p{0.32\linewidth}|p{0.32\linewidth}|p{0.32\linewidth}|}
\hline
\textbf{Regulation Year} & \textbf{Semester} & \textbf{Credits}\\\hline
2021 & III & Lecture: 3 \quad Tutorial: 0 \quad Practical: 0 \quad Credit Points: 3\\\hline
\end{tabular}
\vspace{1.5em}

%=== SCOPE / SUMMARY ==================================================
\begin{tcolorbox}[colback=CorpLightBlue,colframe=CorpBlue,
                  title=Course Overview,
                  fonttitle=\bfseries]
This document presents the curriculum for the \textbf{Engineering Materials} course (Regulation 2021, Semester III). It outlines the objectives, detailed content, learning outcomes, references and supplementary material such as tables and a representative graph.
\end{tcolorbox}

%=== 1. COURSE OBJECTIVES ============================================
\vspace{1.5em}\noindent\textcolor{CorpBlue}{\textbf{\large 1. Course Objectives}}\vspace{0.5em}
\begin{itemize}
    \item To learn the constructing the phase diagram and using of iron‑iron carbide phase diagram for microstructure formation.
    \item To learn selecting and applying various heat treatment processes and its microstructure formation.
    \item To illustrate the different types of ferrous and non‑ferrous alloys and their uses in engineering field.
    \item To illustrate the different polymer, ceramics and composites and their uses in engineering field.
    \item To learn the various testing procedures and failure mechanism in engineering field.
\end{itemize}

%=== 2. COURSE CONTENT ==============================================
\vspace{1.5em}\noindent\textcolor{CorpBlue}{\textbf{\large 2. Course Content}}\vspace{0.5em}
\rowcolors{2}{CorpGrey}{white}
\begin{tabularx}{\linewidth}{>{\raggedright\arraybackslash}p{0.05\linewidth}
                         >{\raggedright\arraybackslash}X
                         >{\centering\arraybackslash}p{0.07\linewidth}}
\rowcolor{CorpBlue}\textcolor{white}{\textbf{No.}} &
\textcolor{white}{\textbf{Topic}} &
\textcolor{white}{\textbf{Credits}}\\
1 & Constitution of alloys – Solid solutions, substitutional and interstitial – phase diagrams, Isomorphous, eutectic, eutectoid, peritectic, and peritectoid reactions, Iron‑Iron carbide equilibrium diagram. Classification of steel and cast‑Iron microstructure, properties and application. & 9\\
2 & Definition – Full annealing, stress relief, recrystallisation and spheroidising – normalizing, hardening and tempering of steel. Isothermal transformation diagrams – cooling curves superimposed on I.T. diagram – continuous cooling Transformation (CCT) diagram – Austempering, Martempering – Hardenability, Jominy end quench test – case hardening, carburizing, Nitriding, cyaniding, carbonitriding – Flame and Induction hardening Vacuum and Plasma hardening – Thermo‑mechanical treatments – elementary ideas on sintering. & 9\\
3 & Effect of alloying additions on steel (Mn, Si, Cr, Mo, Ni, V, Ti \& W) – stainless and tool steels – HSLA – Maraging steels – Grey, white, malleable, spheroidal – alloy cast irons, Copper and its alloys – Brass, Bronze and Cupronickel – Aluminium and its alloys; Al‑Cu – precipitation strengthening treatment – Titanium alloys, Mg‑alloys, Ni‑based – super alloys – shape memory alloys – Properties and Applications – overview of materials standards. & 9\\
4 & Polymers – types of polymers, commodity and engineering polymers – Properties and applications of PE, PP, PS, PVC, PMMA, PET, PC, PA, ABS, PAI, PPO, PPS, PEEK, PTFE, Thermoset polymers Urea and Phenol formaldehydes – Nylon, Engineering Ceramics – Properties and applications of Al\(_2\)O\(_3\), SiC, Si\(_3\)N\(_4\), PSZ and SIALON – intermetallics – Composites – Matrix and reinforcement Materials – applications of Composites – Nano composites. & 9\\
5 & Mechanisms of plastic deformation, slip and twinning – Types of fracture – fracture mechanics – Griffith's theory – Testing of materials under tension, compression and shear loads – Hardness tests (Brinell, Vickers and Rockwell), Micro and nano‑hardness tests, Impact test Izod and Charpy, fatigue and creep failure mechanisms. & 9\\
\end{tabularx}

%=== 3. LEARNING OUTCOMES ===========================================
\vspace{1.5em}\noindent\textcolor{CorpBlue}{\textbf{\large 3. Learning Outcomes}}\vspace{0.5em}
\begin{itemize}
    \item Explain alloys and phase diagram, Iron‑Iron carbon diagram and steel classification.
    \item Explain isothermal transformation, continuous cooling diagrams and different heat treatment processes.
    \item Clarify the effect of alloying elements on ferrous and non‑ferrous metals.
    \item Summarize the properties and applications of non‑metallic materials.
    \item Explain the testing of mechanical properties.
\end{itemize}

%=== 4. REFERENCES ==================================================
\vspace{1.5em}\noindent\textcolor{CorpBlue}{\textbf{\large