\documentclass[11pt,a4paper]{article}
\usepackage[margin=0.75in]{geometry}
\usepackage{tabularx}
\usepackage{multirow}
\usepackage{graphicx}
\usepackage{helvet}
\renewcommand{\familydefault}{\sfdefault}
\usepackage[T1]{fontenc}
\usepackage[utf8]{inputenc}
\usepackage{array}
\setlength\parindent{0pt}
\begin{document}

%--- HEADER -------------------------------------------------
\begin{tabular}{|p{0.20\textwidth}|p{0.55\textwidth}|p{0.25\textwidth}|}
\hline
\centering\textbf{LOGO} &
\centering\textbf{Materials \& Metallurgy Group}\\[0.2em]
\centering\textbf{Material Specification \\[Material Name]} &
\begin{tabular}{@{}l l@{}}
\textbf{Doc. No.} & 12345 \\[0.2em]
\textbf{Sheet}   & Page 1 of 1
\end{tabular} \\
\hline
\end{tabular}

\vspace{1.5em}

%--- 1. SCOPE -----------------------------------------------
\vspace{1em}\noindent\textbf{1. SCOPE:}
\vspace{0.5em}

This specification provides an overview of worldwide steel production trends, highlighting the six largest producers by nation in 2012, with emphasis on China’s leading position. It is intended for reference in material selection and comparative analysis.

%--- 2. SPECIFICATIONS ---------------------------------------
\vspace{1em}\noindent\textbf{2. SPECIFICATIONS:}
\vspace{0.5em}

\begin{tabular}{|c|c|c|c|}
\hline
\textbf{Code} & \textbf{Colour} & \textbf{Equivalent Spec.} & \textbf{As Per Standard} \\
\hline
N/A & N/A & N/A & N/A \\
\hline
\end{tabular}

%--- 3. CHEMICAL COMPOSITION -------------------------------
\vspace{1em}\noindent\textbf{3. CHEMICAL COMPOSITION:}
\vspace{0.5em}

\begin{tabular}{|c|c|c|c|c|c|c|c|c|}
\hline
\textbf{C} & \textbf{Si} & \textbf{Mn} & \textbf{P} & \textbf{S} & \textbf{Cr} & \textbf{Ni} & \textbf{Mo} & \textbf{Others} \\
\hline
\textit{Minimum} & – & – & – & – & – & – & – & – \\
\hline
\textit{Maximum} & – & – & – & – & – & – & – & – \\
\hline
\end{tabular}

%--- 4. MECHANICAL PROPERTIES -------------------------------
\vspace{1em}\noindent\textbf{4. MECHANICAL PROPERTIES:}
\vspace{0.5em}

\begin{tabular}{|c|c|c|}
\hline
\textbf{Property} & \textbf{Minimum} & \textbf{Maximum} \\
\hline
Hardness as Supplied & – & – \\
\hline
\end{tabular}

%--- 5. METALLURGICAL PROPERTIES ----------------------------
\vspace{1em}\noindent\textbf{5. METALLURGICAL PROPERTIES:}
\vspace{0.5em}

\begin{tabular}{|p{0.15\textwidth}|p{0.78\textwidth}|}
\hline
\textbf{5.1 Austenite grain size} & Not applicable / data not provided. \\
\hline
\textbf{5.2 Inclusion rating (Thin)} & Not applicable / data not provided. \\
\hline
\textbf{5.3 Inclusion rating (Thick)} & Not applicable / data not provided. \\
\hline
\textbf{5.4 Other metallurgical notes} & No additional metallurgical data supplied in source material. \\
\hline
\end{tabular}

\end{document}